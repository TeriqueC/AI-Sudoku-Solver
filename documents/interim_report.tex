\documentclass[]{final_report}
\usepackage{graphicx}
\usepackage{hyperref}
\usepackage{fix-cm}


%%%%%%%%%%%%%%%%%%%%%%
%%% Input project details
\def\studentname{Terique Carnegie}
\def\reportyear{2024}
\def\projecttitle{Playing Games and Solving Puzzles Using Artificial Intelligence}
\def\supervisorname{Michail Fasoulakis}
\def\degree{BSc (Hons) in Computer Science}
\def\fullOrHalfUnit{Full Unit} % indicate if you are doing the project as a Full Unit or Half Unit
\def\finalOrInterim{Interim Report} % indicate if this document is your Final Report or Interim Report

\begin{document}

\maketitle

%%%%%%%%%%%%%%%%%%%%%%
%%% Declaration

\chapter*{Declaration}

This report has been prepared on the basis of my own work. Where other published and unpublished source materials have been used, these have been acknowledged.

\vskip3em

Word Count: 

\vskip3em

Student Name: \studentname

\vskip3em

Date of Submission: 

\vskip3em

Signature: \studentname

\newpage

%%%%%%%%%%%%%%%%%%%%%%
%%% Table of Contents
\tableofcontents\pdfbookmark[0]{Table of Contents}{toc}\newpage

%%%%%%%%%%%%%%%%%%%%%%
%%% Your Abstract here

\begin{abstract}

  The general purpose of artificial intelligence is to give computers the ability to replicate human intelligence through the use of problem-solving algorithms. AI has the potential to outperform human intelligence in solving many different and difficult problems due to its ability to “think” by running problem solving computations at much greater speeds and accuracy than a human and it does this through leveraging the superior processing capabilities of a computer~\cite{AIforSocialGood}. 

 

  Due to the nature of problem-solving algorithms, Artificial Intelligence is able to provide assistance and insight into many different fields of expertise making it popular and very much in demand as an asset to humanity. In order to increase the value of AI, it is necessary to deeply research ways to improve problem solving algorithms and to discover which of the algorithms works best for a specific problem, and it is also important to develop methods for humans and AI to safely interact. One way this can be achieved is through games and puzzles. Using games and puzzles as a means to research and further develop AI can be very beneficial as it allows for problem solving algorithms to be tested against fun solvable challenges which can provide insight on how well the algorithms are able to solve problems, using games can also provide a way for humans to interact with artificial intelligence through either testing the speed and efficiency of both the player and the algorithms or through the use of 2 player games like chess and checkers. This type of approach to using artificial intelligence has been seen in the past from the first working checkers program to appear in 1952~\cite{Kister1957}, and chess playing programs being developed shortly thereafter~\cite{Strachey1952} which had a role in furthering research and development of modern algorithms and uses of Artificial Intelligence. 

\end{abstract}
\newpage

%%%%%%%%%%%%%%%%%%%%%%
%%% Project Spec

\chapter*{Project Specification}
\addcontentsline{toc}{chapter}{Project Specification}
Your project specification goes here.

%%%%%%%%%%%%%%%%%%%%%%
%%% Introduction
\chapter{Introduction}

\section{Artificial Intelligence Solving a Sudoku}

\section{Purpose of my Application}

\section{Project Milestones}

%%%%%%%%%%%%%%%%%%%%%%
%%% Application Development
\chapter{Application Development}

\section{Development of Terminal Interface}

\section{Development of Backtracking Algorithm}
\subsection{What is Backtracking?}
\subsection{Implementation of Backtracking Algorithm}

\section{Development of Graphical User Interface}

If you wish to print a short excerpt of your source code,  ensure that you are using a fixed-width sans-serif font such as the Courier font. By using the \verb|verbatim| environment your code will be properly indented and will appear as follows:

\begin{verbatim}
static public void main(String[] args) {
  try  {
    UIManager.setLookAndFeel(UIManager.getSystemLookAndFeelClassName());
  }
  catch(Exception e) {
    e.printStackTrace();
  }
  new WelcomeApp();
} 
\end{verbatim}

%%%%%%%%%%%%%%%%%%%%%%
%%% Current Progress
\chapter{Current Progress}

\section{Application Architecture}
\section{Running the Application}

%%%%%%%%%%%%%%%%%%%%%%
%%% Next Obectives
\chapter{Next Objectives}

\section{Development of Neural Network solution}
\subsection{What is a Neural Network?}
\subsection{How can this be Implemented into my Application?}

\section{Further Improvments planned for the GUI}

Most final reports will contain a mixture of figures and charts along with the main body of text. The figure caption should appear directly after the figure as seen in Figure~\ref{fig:logo} whereas a table caption should appear directly above the table. Figures, charts and tables should always be centered horizontally. 

\begin{figure}[h]
\centering
\fboxsep 2mm
\framebox{
	\includegraphics[width=6cm]{logo} 
}
\caption{\label{fig:logo} Logo of RHUL.}
\end{figure} 

%%%%%%%%%%%%%%%%%%%%%%
%%% conclusion
\chapter{Conclusion}


%%%% ADD YOUR BIBLIOGRAPHY HERE
\newpage
\begin{thebibliography}{99}
\addcontentsline{toc}{chapter}{Bibliography}
\bibitem{AIforSocialGood} 
AI for Social Good (n.d.). Artificial Intelligence vs. Human Intelligence: Exploring the Debate and Key Points. [online] Available at: \url{https://aiforsocialgood.ca/blog/artificial-intelligence-vs-human-intelligence-exploring-the-debate-and-key-points} [Accessed 8 Oct. 2024].
\bibitem{Strachey1952} C. Strachey, Logical or non-mathematical programmes, in: Proc. Association for Computing Machinery Meeting, Toronto, ON, 1952, pp. 46–49. 
\bibitem{Kister1957} J. Kister, P. Stein, S. Ulam, W. Walden, M. Wells, Experiments in chess, J. ACM 4 (1957) 174–177.
\bibitem{GeeksforGeeks} GeeksforGeeks (n.d.). Sudoku | Backtracking-7. [online] Available at: \url{https://www.geeksforgeeks.org/sudoku-backtracking-7/} [Accessed 9 Oct. 2024]. 
\bibitem{Simonis} Simonis, H., n.d. Constraint Programming in Action. [pdf] Available at: \url{https://ai.dmi.unibas.ch/_files/teaching/fs21/ai/material/ai26-simonis-cp2005ws.pdf} [Accessed 10 Oct. 2024]. 
\bibitem{AkinDavid} Charles Akin-David, Richard Mantey, n.d. Solving Sudoku with Neural Networks. [pdf] Available at: \url{https://cs230.stanford.edu/files_winter_2018/projects/6939771.pdf} [Accessed 10 Oct. 2024].

\end{thebibliography}
\label{endpage}



\end{document}

\end{article}
